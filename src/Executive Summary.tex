\chapter*{Executive Summary}
\addcontentsline{toc}{chapter}{Executive Summary}

The project \emph{Research on Numerical Simulation of Downhole Drill String Dynamics} reviewed a set of open-source and published models to assess their features.  Of primary interest were models rated to drill string dynamics.  Two models were selected for evaluation.  While not evaluated, a thorough review of a family of bit models was conducted as it will be of interest in the future.

A series of Test Cases was devised to compare the models.  A set of procedures and methods for evaluating drill string models was established and documented.  Each model was then run on the Test Cases.

The first model selected was the model from Aarsnes and Shor. Two versions of the model are available, a more mature version in Matlab and an upgraded version written in Python. While the Python version is being written to improve the older version, development is still in progress. Currently, the model only simulates the torsional motion of the drill string and the axial motion will be added to the Python version. It is a distributed model in that the governing equations of torsional motion are solved by finite difference method. The model assumes constant torque on bit during drilling. Friction is modeled by the Coulomb friction and viscous damping. Coulomb friction is modeled as a jump, where the dynamic friction is calculated as a certain ratio of static friction. Also transition from static to dynamic friction occurs when the torsional velocity overcomes the critical velocity. 

The second model selected was developed at ExxonMobil and is written in Python. The model is a coupled axial-torsional system that rigorously accounts for the interactions between various components of the drill string. This allows for thorough examinations into the system's behavior under diverse operational settings. The precise representation of axial and tangential forces, as well as the explanation of stick-slip reasons, is achieved in the model through the integration of friction models and the bit-rock interaction model. The system incorporates 2 degrees of freedom per node, encompassing both axial and rotational displacements as well as velocities. The governing equations are solved using the ODE45 solver and subsequently verified by comparison with field data. Significantly, the model considers the trajectory of the well, the effects of buoyancy, and uniform mud drag in order to accurately calculate the friction force. Its versatility extends to simulating Mud-motor and heave compensator dynamics. Additionally, the model's Python-based source code is designed with modularity, ensuring ease of use and offering comprehensive guidance for effective utilization. 

The simulation results from both models for all test cases were compared and evaluated. The results from both model matched well showing similar behavior of angular velocity and torque on top drive and bit. Specifically both model showed similar amplitude and frequency of the drill string vibration, Also showing similar response during stick-slip event.

The availability of these well-structured Test Cases allows for rigorous comparison and verification of drill string models, enabling researchers and engineers to assess their model's accuracy and performance in various wellbore scenarios. The data and findings obtained from these Test Cases offer valuable insights and constitute a substantial contribution to the field of drill string modeling and simulation. As a result, this study serves as a valuable reference for further research and development in the domain of drilling engineering and related disciplines. 
