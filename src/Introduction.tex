\chapter{Introduction}
This document was created as part of the project \emph{Research on Numerical Simulation of Downhole Drill String Dynamics}, a fundamentals research project at CNPC USA.  The project was proposed and kicked off in 2023.

The concept for a model review and evaluation result from conversations with members of the \osdc{} (\referencename~\cite{ref:pastusek2019a}).  A number of models have been released to the community's \emph{Github} site, however, it was not known exactly how the models compare in terms of features and performance.  At the same time, CNPC was both interested in studying drill string models and is a member of two industry-university research programs.  These programs may provide access to drilling dynamic models.  Based on this it was decided to review the drilling dynamic models on the \osdc{} as well as others that may be accessible through other industry memberships.  Promising drill string models would then be selected for evaluation and testing.

\section{Objectives}
The primary motivation for CNPC USA was to evaluate soft string models.  However, in order to compare models, it is necessary to have a set of Test Cases to use as a benchmark.  In addition, well defined procedures should be used in order to allow a direction comparison.  Based on these requirements, the following were adopted as objectives.
\begin{numberedlist}
	\item \emph{Review} all the models that were potential available to determine their features.
	\item \emph{Establish} a set Test Cases for comparing drill string models.
	\item \emph{Document} procedures and methods used in the comparisons.
	\item \emph{Evaluate} soft string models and assess their features and performance.
\end{numberedlist}

\section{Organization of this Document}
\notfinished{}
%This document is organized into several parts.  In \partname~\ref{prt:review} is contains the preliminary information of the model review and test cases.  Each of these \chaptername{}s is intended to be stand alone.  In addition, the Test Cases are generalized and not specific to the particular models evaluated.  They can be used to compare any drill string models.
%
%In \partname~\ref{prt:evaluation} two models are evaluated in-depth and run on the Test Cases.  Each model contains a dedicated \chaptername{} and then the results are directly compared in a separate \chaptername{}.
%
%\partname{}~\ref{prt:finalremarks} contains conclusions and recommendations.
%
%\partname{}~\ref{prt:backmatter} contains additional information in Appendices.  It also contains reference material such as an acronym reference, a glossary, and a bibliography.
%

\noindent{}\reviewcomment{Move \emph{Model Review} and \emph{Models Select} to \emph{Organization}.  Keep \emph{Important Notes...}}

\section{Model Review}
The models associated with drilling dynamics that were available on the \osdc{} Github site, as well as those potentially available through industry-university research programs were reviewed.  However, not all of these models were exercised.  Some of the models did not pertain directly to drill strings, lacked key features, or the source code was unavailable.  For a summary of all the models reviewed, please see \chaptername~\ref{ch:modelreview}.

\section{Model Evaluation}
\subsection{Models Selected}
Two models were selected for in-depth evaluation.  The first is the torsional model of Aarsnes-Shor.  The model has two versions available; the more mature Matlab version and a newer port to Python.  The second model is from ExxonMobil and written in Python.

\subsection{Important Notes on the Models}
It should be noted that these models are considered research models and are currently still in development.  As such, it was expected to lack some features a commercial code would have.  In addition, some issues were expected to be encountered.  Any issues are reported solely in the interest of helping to improve the models and providing important feedback to the developers.  \important{At no time should any comments in this report be construed as criticisms of these models.}

On the contrary, this collaboration is viewed as demonstrating the benefits of open-source software.  A wider audience gets to make use of the software and access to the source code means issues can examined, tested, and feedback provided to the developers.

%The team is eternally grateful for allowing access to these models.
\section{Nomenclature}
\notfinished{}
\reviewcomment{Add info about \emph{transient} and \emph{steady-state} use.}


\section{Guidance and Technical Committee}
Several members of the industry and academia played key roles in this project.  Their support and guidance is greatly appreciated.  Without their assistance, this project would not have been possible.  Brief biographies are presented below.

\begin{committeemember}{Rajat Dixit}{MS from Indian Institute of Technology}{Wells Engineer (Drilling) at ExxonMobil}
Expertise includes fluids and thermal sciences, FEA, computational sciences, solid mechanics and heat transfer.
\end{committeemember}

\begin{committeemember}{Eduardo Gildin}{Ph.D. from University of Texas}{Professor at Texas A\&M}
Research interests include model reduction of large-scale dynamical systems, control and optimization of large-scale dynamical systems and reservoir modeling and simulation
\end{committeemember}

\begin{committeemember}{Anirban Manna}{MS from Indian Institute of Technology}{Wells Research Engineer at ExxonMobil}
Expertise includes FEA, fatigue, buckling, data analysis and drilling dynamics.
\end{committeemember}

\begin{committeemember}{Paul Pastusek}{MBA from University of Houston}{Drilling Mechanics Advisor at ExxonMobil}
Areas of interest include rig automation, debottlenecking, forensics analysis, drilling dynamics, borehole quality and technical training
\end{committeemember}

\begin{committeemember}{Greg Payette}{Ph.D. from Texas A\&M}{Wells Research Engineer at ExxonMobil}
Research interests include drilling dynamics, wellbore quality, open-source software, FEA and drilling automation
\end{committeemember}

\begin{committeemember}{Roman Shor}{Ph.D. from University of Texas}{Associate Professor at University of Calgary}
Research interests include in the areas of drill string dynamics modelling and control, drilling optimization and drilling systems automation.
\end{committeemember}





