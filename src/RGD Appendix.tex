\chapter{RGB Model Work Flow}
\label{ap:rgbworkflow}
%\section{Mathematical background of RGD model}
This appendix briefly summarizes the work flow of RGB model written in Matlab. A more detailed description can be seen in \referencename~\cite{ref:zhang2020a}. This code is for a symmetric full blade which is defined as a D1 bit in the paper. The code calculates coupled PDE and ODE by solving the matrix
\begin{equation}\label{matrix}
  \bm{Adx} + \bm{BX} = \bm{Q}
\end{equation}
where $\bm{dx}$ and $\bm{X}$ are defined as follow:

\vspace{\abovedisplayskip}
\noindent\begin{minipage}{.4\linewidth}
	\begin{equation}
		\bm{x}=
		\begin{bmatrix}
		\dot{u} \\
		u \\
		\psi \\
		\dot{\psi} \\
		a_1\ \\
		\vdots \\
		a_N \\
		\end{bmatrix}
	\end{equation}
\end{minipage}%
\hfill
\begin{minipage}{.4\linewidth}
	\begin{equation}
		\bm{dx}=
		\begin{bmatrix}
		\ddot{u} \\
		\dot{u} \\
		\dot{\psi} \\
		\ddot{\psi} \\
		\dot{a_1}\ \\
		\vdots \\
		\dot{a_N} \\
		\end{bmatrix}
	\end{equation}
\end{minipage}
\vspace{\belowdisplayskip}

The bit trajectory function, which is $\overline{h}(\tau, \theta$), is introduced and PDE can be derived from equation
\begin{equation}\label{PDE}
\frac{\partial \overline{h}}{\partial \tau} + (w_0 + \dot{\psi})\frac{\partial \overline{h}}{\partial \theta}-v_0 = 0
\end{equation}
where the equation (\ref{PDE}) is solved by Galerkin-method which approximates $\overline{h}(\tau, \theta$) by using base functions $a_1, a_2, ..., a_N$ (N=25 as default), where $\tau$ is scaled time and $\theta$ is rotation angle of the bit blade. The base functions can be obtained by minimizing the residual which are shown below (please note that all the derivations are for D1-bit):
\begin{equation}\label{GM}
\begin{split}
R &= \left(\frac{n \theta}{2 \pi}-1\right)\dot{u}_b + \frac{n \theta}{2 \pi}\dot{a}_1 + \sum_{k=1}^{N-1}\dot{a}_{k+1}sin\left(\frac{nk\theta}{2}\right) \\ + &(w_0 + \dot{\psi}_0)\left[u_b\frac{n}{2\pi}+a_1\frac{n}{2\pi} + \sum_{k=1}^{N-1}\frac{a_{k+1}nk}{2}cos\left(\frac{nk\theta}{2}\right)\right]
\end{split}
\end{equation}
where n is the number of bit blade. In the paper the blade number is divided into $n_i$, and $n_o$ that are number of inner blades and outer blades, respectively. However, the code runs for D1 bit and it contains n instead of $n_i and n_o$.

Minimizing $R(\theta, \tau)$ can be achieved by making it orthogonal to the base functions over the domain $\theta \in \left[0, \frac{2\pi}{n_i}\right]$, which results in
\begin{equation}\label{GM1}
 \int_{0}^{\frac{2\pi}{n}}R(\theta,\tau)\frac{n\theta}{2\pi}d\theta = 0
\end{equation}
and
\begin{equation}\label{GM2}
 \int_{0}^{\frac{2\pi}{n}}R(\theta, \tau)sin\left(\frac{n_im\theta}{2}\right)d\theta=0, \;\; m= 1,...,N-1
\end{equation}
where N is the number of Galerkin base functions.

Equation (\ref{GM1})-(\ref{GM2}) represent a system of $N^{th}$ first order equations for $a_i,\; i=1,2,...,N$ that is
\begin{equation}\label{Norder}
  \dot{a}_i = \Phi_i(\dot{u}, u, a_1, ..., a_N), i=1,...,N.
\end{equation}
and also coupled with the equation of motion which are
\begin{equation}\label{GE1_}
  \ddot{u} = \psi(\varpi_0 - \varpi)
\end{equation}
\begin{equation}\label{GE2_}
  \ddot{\phi} + \phi = \Gamma_0 - \Gamma
\end{equation}
where
\begin{equation}\label{GE1}
  \varpi-\varpi_0 = n(\delta_n - \lambda(g(\nu)) - \frac{2\pi v}{w_0}
\end{equation}
\begin{equation}\label{GE2}
  \Gamma-\Gamma_0 = n(\delta_n -\beta \lambda(g(\nu)) - \frac{2\pi v}{w_0}
\end{equation}
and $\delta_n$ is given as
\begin{equation}\label{deltan1}
  \delta_n = \overline{h}\left(\frac{2\pi}{n}, n\right) + u(\tau)
\end{equation}
where $\varpi$, $\Gamma$ represent scaled WOB, TOB, respectively,  $\beta$ is number characterizing bit/rock interaction (generally less than 1). $v$ and $w$ are the scaled axial and angular velocity, respectively that are
\begin{equation}\label{scaled_axial_ve}
  v = v_0 + \dot{u}
\end{equation}
\begin{equation}\label{scaled_angular_vel}
  w = w_0 + \dot{\psi}
\end{equation}
so far we have N+2 equations with N+4 unknowns ($u, \dot(u), \psi, \dot(\psi), a_1, ..., a_N$). and two additional equations are obtained from relationship below:
\begin{equation}\label{axial_dis_vel}
  \dot{u} = \frac{\partial u}{\partial \tau}
\end{equation}
\begin{equation}\label{angular_dis_vel}
  \dot{\psi} = \frac{\partial \psi}{\partial \tau}
\end{equation}
Finally we obtained N+4 equations for the same number of unknowns and represent as linear from of $\bm{Adx} + \bm{BX} = Q$. and solve the equation for given time interval. However, the discontinuity should be investigated every time step which affects the boundary condition from rock-bit interaction. The model classifies the discontinuity to three which are 1) normal drilling, 2) axial stick, and 3) bit bonce. Below summarizes the conditions for each mode:
\begin{equation}\label{drillingmodes}
  \begin{cases}
    Normal\,drilling, & \mbox{if $\delta_n > 0, w > 0$, and $v > 0$ }  \\
    Axial\, stick, & \mbox{if $\delta_n >0, w > 0$, and $v = 0$ } \\
    Bit\,bounce, & \mbox{if $\delta_n = 0$, and $w > 0$}
  \end{cases}
\end{equation}

For example, normal drilling mode simplifies the equations (\ref{GE1}), and (\ref{GE2}) since g($\nu$) = 0 which can be represented as
\begin{equation}\label{GE1_normaldrilling}
  \varpi-\varpi_0 = n\left(\delta_n - \frac{2\pi v}{w_0}\right)
\end{equation}
\begin{equation}\label{GE2_normaldrilling}
  \Gamma-\Gamma_0 = n\left(\delta_n - \frac{2\pi v}{w_0}\right)
\end{equation}
and also, the depth of cut (\ref{deltan1}) is reduced to
\begin{equation}\label{deltan_normaldrilling}
  \delta_n = a_1 + u(\tau)
\end{equation}
Therefore, equation (\ref{matrix}) can be solved with reduced equations from equations (\ref{GE1_normaldrilling}) - (\ref{deltan_normaldrilling}).
