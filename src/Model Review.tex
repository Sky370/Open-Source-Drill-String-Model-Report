\chapter{Model Review}
\label{ch:modelreview}

Below is a brief review of the models available or referenced on the \osdc{} Github site related to drilling dynamics.  Some key features are noted as well as reasons for selecting or not selecting the model for further evaluation.

\section{Wilson}
The basis for the model is the Ph.D. dissertation of Joshua Wilson (\referencename~\cite{ref:wilson2017a}).  The model was further developed at Scientific Drilling.  It is a full featured stiff-string model based on the finite element method.  It includes coupled flexibility of the drill string, geometric nonlinearity, wellbore contact, fluid mass, and damping from fluid, amongst other features.

The full featured model is not open-source.  Instead, a limited version has been released to the \osdc{}.  Specifically, it has the following limitations.
\begin{bulletedlist}
	\item Straight wellbores only
	\item Constant wellbore OD
	\item No buckling
	\item No dynamic calculations
	\item Graphic user interfaces have been removed
	\item Advanced plotting features have been removed
\end{bulletedlist}
Specifically, since the dynamics have been removed, this model was not considered fit for purpose.

\section{Texas A\&M Model}
Texas A\&M developed a model to serve as simulator used for training horizontal directional drilling (\referencename~\cite{ref:losoya2020a}).  It was developed on top of a high-performance open-source library written in C++.  The model seems promising; it incorporates multibody physics, CFD, and other solvers as well as leveraging GPU acceleration.  However, the full model has not been released.  Importantly for our purposes, the dynamics portion has not been open-sourced.  Therefore, this model was not selected.

\section{Tulsa University Models}
Two models from Tulsa University were evaluated.  The source code for these models is available to CNPC USA through the TUDRP program. Ultimately, neither were selected for further evaluation.  The reviews of each and reasoning for not selecting them are presented below.

\subsection{Lateral Whirl Model}
The Ph.D. dissertation of Kriti Singh (\referencename~\cite{ref:singh2019a}) presents a model for lateral dynamics of drill strings.  The primarily used to predict whirl in the drill string.  Dynamics are calculated in terms of Euler-Bernoulli beam elements and the solution is the general solution to the differential equation.  The Dynamic Transfer Matrix Simulator can be used to calculate the response of the BHA assembly to the specified excitation, for the specified RPM, WOB, and other operating parameters.  For the Dynamics Transfer Matrix Simulator it is assumed that if the deflection is larger than the borehole clearance, backward whirl will occur.  The numerical calculation is based on a single mass rotation disk.  This would only allow capturing the first natural frequency.  Since this model is a single degree of freedom lateral model, it does not fit the criteria of this project and it was not selected.

\subsection{Soft String Model}
The soft string model from Tulsa University is originally based on \referencename~\cite{ref:miska2015a} where the model was originally developed.  This model is ``dynamic'' in the sense that it incorporates inertia effects.  This is an improvement on other models which are purely static.  A key assumption is that the pipe is not rotating, therefore, torsional oscillations are ignored.  Essentially, this becomes dynamics of a rigid body in the axial direction while accounting for the varied frictional effects (from string orientation) along the string.  Since this project was investigating axial and torsional dynamic vibration models, this model was not suitable.

In \referencename~\cite{ref:zamanipour2018a}, the model from  \referencename~\cite{ref:miska2015a} is imporoved to add axial stiffness, static friction, and drilling fluid drag.  The primary objective is calculation of forces for drillstrings during tripping operations.  It uses a lumped mass and spring system.  It claims to be an axial ``stiff string'' model.  It allows for the string to be at the top or bottom of the hole automatically, however, it does not seem to account for the no contact case.  The wellbore is considered to be 2D.  A comparison is done between the axial ``stiff string'' model and a soft string model.  It seems that ``dynamic'' and ``static'' would be better nomenclature for these models.  The improved model does seem to be an axially dynamic model.  However, torsional effects are still not considered.  Therefore, this model was not selected.

\section{Richard-Germay-Detournay Bit Model}
While the current project focuses on drill string models, it is natural to couple a string model with a bit model. The Richard-Germay-Detournay family of bit models (RGD models) represents theoretical bit-rock interactions.  Moreover, they represent a more sophisticated approach than many bit models.  Therefore, a review of these models is included.  Please see \chaptername~\ref{ch:rgdmodels} for a complete review.

\section{Aarsnes-Shor Models}
The Aarsnes-Shor model is a soft string torsional model.  It was selected for evaluation.  Please see \chaptername~\ref{ch:aarnessshormodel} for a complete analysis.

\section{ExxonMobil Model}
The ExxonMobil model is a soft string axial-torsional model.  It was selected for evaluation.  Please see \chaptername~\ref{ch:exxonmobilmodel} for a complete analysis.