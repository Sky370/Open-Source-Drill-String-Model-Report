% GLOSSARY ENTRIES.

%\addcontentsline{toc}{chapter}{Glossary}
%\newcommand*{\gsep}{\hspace*{1.5pt}:}
%\newcommand*{\gsep}{.}

% The \glossary command is comma separated, so use this command to add commas to the glossary.
\newcommand*{\comma}{,}

%\newglossaryentry{}
%{
%	name=,
%	description={}
%}

\newglossaryentry{coupled}
{
	name=coupled,
	description=In classical mechanics two variables are coupled if they appeared in each others' equations of motion.  This could be thought of as one variable \textit{always} influences the other
}

\newglossaryentry{coupledcode}
{
	name=coupled code,
	description=In the drilling dynamics realm\comma this taken the definition of a dynamic drill string code combined with a high fidelity drill bit code
}

\newcommand*{\dynamicforcebalance}{$a \neq 0; \sum F = MA$}
\newglossaryentry{dynamicmodel}
{
	name=dynamic model,
	description=A model that deals with the branch of mechanics concerned with the motion of bodies under the action of forces (\dynamicforcebalance)
}

\newglossaryentry{frequencydomainmodel}
{
	name=frequency domain model,
	description=The frequency domain refers to the analysis of mathematical functions or signals with respect to frequency\comma rather than time.  It may be used to refer to graphs plotted with respect to frequency or to a static model that is used for eigenmode and eigenvalue calculations
}

\newglossaryentry{qualitative}
{
	name=qualitative,
	description={Data that is non numeric in nature and cannot be measured}
}

\newglossaryentry{quantitative}
{
	name=quantitative,
	description={Data that is numerical in nature and can be measured and can be classified into two: discrete and continuous}
}

\newglossaryentry{sample}
{
	name=sample,
	description={A selection of observations from a population.  Example: People (or IP addresses) who visited a website on a specific day}
}

\newcommand*{\staticforcebalance}{$a = 0; \sum F = 0$}
\newglossaryentry{staticmodel}
{
	name=static model,
	description=Models that calculate statics.  Statics is the branch of classical mechanics that is concerned with the analysis of force and torque (also called moment) acting on physical systems that do not experience an acceleration (\staticforcebalance) \comma but rather\comma are in static equilibrium with their environment.  Note that this does not mean they are not moving
}

\newglossaryentry{statistic}
{
	name=statistic,
	description={A numerical value associated with an observed sample.  Example: The average amount of time people spent on a website on a specific day}
}

\newglossaryentry{steadystate}
{
	name=steady state,
	description=S
}

\newglossaryentry{steadystatemodel}
{
	name=steady-state model,
	description=This term should not be used.  The preferred terms are \textit{static model} or \textit{dynamic model}.  Steady-state is a condition that does not change in time.  Note that the condition does not change; that does not mean there are no accelerations.  For example\comma the stick-slip of a drill string is steady state
}

\newglossaryentry{supervisedlearning}
{
    name=supervised learning,
    description=Supervised learning is when you already know the label (value) of the target variable.  It is of two types: regression (for continuous variables) and classification (for categorical or discrete values)
}

\newglossaryentry{timedomainmodel}
{
	name=time domain model,
	description=Time domain and dynamic model are synonyms.  See \emph{dynamic model}
}

\newglossaryentry{torquefunction}
{
	name=torque function,
	description=A mathematical equation to convert input parameters such as WOB to a torque value.  Used at the end of a drill string model to generate the torque that would result from the bit-rock interaction.  It does not account for bottom hole patterns or interaction
}

\newglossaryentry{transientmodel}
{
	name=transient model,
	description=This term should not be used.  The preferred term is \textit{dynamic model}.  A transient phase is the pattern of change as a system moves from one equilibrium state to another.  For example\comma the phase between backward whirl and forward whirl
}

\newglossaryentry{uniformdistribution}
{
	name=uniform distribution,
	description=The probability is the same across a range of values (the probability of occurrence is uniformly distributed).  Does not favor a particular outcome
}

\newglossaryentry{unstablemodels}
{
    name=unstable (models),
    description=Models that are very sensitive to small changes in the data (small changes in data lead to a different model)
}

\newglossaryentry{unsupervisedlearning}
{
    name=unsupervised learning,
	description=Unsupervised learning uses machine learning algorithms to analyze and cluster unlabeled data sets.  These algorithms discover hidden patterns in data without the need for human intervention (hence\comma they are ``unsupervised'').  Unsupervised learning models are used for three main tasks: clustering\comma association and dimensionality reduction
}