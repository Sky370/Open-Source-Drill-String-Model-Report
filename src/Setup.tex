% HYPERLINKS	
\usepackage[pdfauthor={Lance A. Endres}, pdftitle={Drilling Dynamics and Mechanics}]{lehyperlink}


% NOTATIONS AND MARKUP
\usepackage{lemarkup}
\usemarkup{}


% SPECIAL NAMES
\newcommand*{\osdc}{\emph{Open-Source Drilling Community}}


% FOOTNOTES
% Change to foot note diving line.  Make it shorter (narrower) and lighter in color.
\usepackage{color}
\definecolor{lightgrey}{rgb}{0.5,0.5,0.5}
\renewcommand{\footnoterule}{%
  \kern -3pt
  {\color{lightgrey}\hrule width 0.15\textwidth height 1pt}
  \kern 2pt
}


% GRAPHICS/FIGURES
\usepackage{legraphicextensions}
\usepackage{subcaption}
\captionsetup[sub]{font=normalsize}

% Set the search path for figures.  This way we can change it if required without having to modified all of the \includegraphics calls.
\graphicspath{{Figures/}}


% TABLES
\usepackage{letable}

% Use tabularx to have adjustable width columns.
\usepackage{tabularx}
% Create a tabularx column with ragged right interiors.  The default paragraph style columns tries to justify the
% text and ends up creating underfull warnings.
\newcolumntype{L}{>{\raggedright\arraybackslash}X}

% Test Case tables.
% Tabularx does not like being split into the beginning and ending arguments of a regular environment, so use this different environment definition.
\NewDocumentEnvironment{testcasetable}{+b}
{
	\begin{tabularx}{\linewidth}{|c|c|c|L|}
		\hline
		\tablecolumnheadervlinesone{Parameter} & \tablecolumnheadervlinestwo{Value}            & \tablecolumnheadervlinestwo{Value}           & \tablecolumnheadervlinestwo{Description} \\[-7pt]
		                                       & \tablecolumnheadervlinestwo{(imperial units)} &  \tablecolumnheadervlinestwo{(metric units)} & \\
		\hline
		#1
	\end{tabularx}
}
{}

\newenvironment{testcaseresulttable}
{
    \begin{tabular}{|c|c|c|c|c|}
        \hline
        \tablecolumnheadervlinesone{Test Cases} & \tablecolumnheadervlinestwo{Frequency}  & \tablecolumnheadervlinestwo{Max TD Torque}  & \tablecolumnheadervlinestwo{Max Bit Velocity}  & \tablecolumnheadervlinestwo{Simulation Time} \\[-7pt]

        \tablecolumnheadervlinesone{} & \tablecolumnheadervlinestwo{($Hz$)} & \tablecolumnheadervlinestwo{($lb\mbox{-}ft$)} & \tablecolumnheadervlinestwo{($RPM$)} &  \tablecolumnheadervlinestwo{($s$)} \\
        \hline
}
{
	\end{tabular}
}

% Package from DY for table
\usepackage{multirow}
\usepackage{makecell}


% FONTS
\usepackage{times}


% LISTS
\usepackage{lelists}
\setlength{\listtopsep}{0pt}
\setlength{\leftlistindent}{0pt}
\setlength{\rightlistindent}{0pt}


% LIST OF NOTATIONS
\usepackage{lelistofnotations}
\spacingbetweennotations{0.5\baselineskip}
\notationheadinglinesize{16pt}


% GLOSSARIES
% Optional arguments are:
%		toc: Adds the main glossary to the table of contents.  The others are added by default.
%		nonumberlist: Prevents the page numbers from being added to each entry in the glossaries.
\usepackage[toc=true, nonumberlist=true, acronym]{glossaries}
\newcommand*{\addacronym}[2]{\newacronym{#1}{\MakeUppercase{#1}}{#2}}
\newcommand*{\addacronymlowercase}[2]{\newacronym{#1}{\MakeLowercase{#1}}{#2}}
\makeglossaries{}


% EPIGRAPHS
% Quotes at the start of chapters.
\usepackage{epigraph}
\renewcommand{\epigraphflush}{center}
\renewcommand{\epigraphrule}{0pt}
\setlength{\epigraphwidth}{\textwidth-3in}
\newcommand*{\formattedepigraph}[2]{\epigraph{\textit{#1}}{--- #2}}


% MATH
\usepackage{amsmath}
\allowdisplaybreaks{}
\usepackage{lemath}
\setlength{\mathwhererightmargin}{0pt}
\setlength{\mathdefspacelength}{5.0pt}
\setlength{\mathwhereparsep}{0ex plus1pt minus1pt}           %DY uncomment
\usepackage{bm}


% CODE
\usepackage{lecode}
\codefontsize{\small}

% SPECIAL FORMATTING
\usepackage{leparskip}
\globalalternateparskip{0.5\baselineskip plus 2pt minus 2pt}

% To use hanging indent (hang) inside of the environment.
\usepackage{hang}
\setlength{\hangingindent}{\parindent}

% Committee member biography formatting.
\newfont{\helveticbc}{phvbc7t at 12pt}
\newenvironment{committeemember}[3]
{
	\pushparskip{}
	\begin{hangingpar}
	{\helveticbc #1}\\%
	\textbf{#2}\\
	\textit{#3}\\
}
{
	\end{hangingpar}
	% No need to pop the parskip.  The change only remains in effect inside of the environment.
}

% Definitions.
\newenvironment{definition}[1]
{
	\pushparskip{}
	\begin{hangingpar}
	{\bfseries#1}\hspace*{0.5em}%
}
{
	\end{hangingpar}
	% No need to pop the parskip.  The change only remains in effect inside of the environment.
}

% Formatting for notes on code modifications.
\newenvironment{codemodifications}
{
	\pushparskip[0.15\baselineskip plus 2pt minus 2pt]{}
}
{
	% No need to pop the parskip.  The change only remains in effect inside of the environment.
}

\newenvironment{codemodification}[1]
{\noindent\begin{hangingpar}\textbf{\textcode{\MakeUppercase{Line: }#1}}\\}
{\end{hangingpar}}


% WARNINGS
% Suppress warnings about too many floats on a page.
\usepackage{silence}
\WarningFilter*{latex}{Text page \thepage\space contains only floats}


% INCLUDE A BIBLIOGRAPHY.
\bibliographystyle{leplain} 