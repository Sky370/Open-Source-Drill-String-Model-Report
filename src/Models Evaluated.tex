\chapter{Models Evaluated}

\section{Wilson}
The basis for the model is the Ph.D. dissertation of Joshua Wilson (\referencename~\cite{ref:wilson2017a}).  The model was further developed at Scientific Drilling.  It is a full featured stiff-string model based on the finite element method.  It includes coupled flexibility of the drill string, geometric nonlinearity, wellbore contact, fluid mass, and damping from fluid, amongst other features.

The full featured model is not open-source.  Instead, a limited version has been released to the \osdc{}.  Specifically, it has the following limitations.
\begin{bulletedlist}
	\item Straight wellbores only
	\item Constant wellbore OD
	\item No buckling
	\item No dynamic calculations
	\item Graphic user interfaces have been removed
	\item Advanced plotting features have been removed
\end{bulletedlist}
Specifically, since the dynamics have been removed, this model was not considered fit for purpose.

\section{Tulsa University Models}
Two models from Tulsa University were evaluated.  The source code for these models is available to CNPC USA through the TUDRP program. Ultimately, neither were selected for further evaluation.  The reviews of each and reasoning for not selecting them are presented below.

\subsection{Lateral Whirl Model}
The Ph.D. dissertation of Kriti Singh (\referencename~\cite{ref:singh2019a}) presents a model for lateral dynamics (whirl) of drill strings.

Dynamics are calculated in terms of Euler-Bernoulli beam elements and the solution is the general solution to the differential equation.

The Dynamic Transfer Matrix Simulator can be used to calculate the response of the BHA assembly to the specified excitation, for the specified RPM, WOB, and other operating parameters.

For the Dynamics Transfer Matrix Simulator it is assumed that if the deflection is larger than the borehole clearance, backward whirl will occur.

Primarily used to predict whirl in the drill string.

The numerical calculation is based on a single mass rotation disk.  This would only allow capturing the first natural frequency.

\notfinished{}


\subsection{Soft String Model}
The soft string model from Tulsa University is originally based on \referencename~\cite{ref:miska2015a} where the model was originally developed.  This model is ``dynamic'' in the sense that it incorporates inertia effects.  This is an improvement on other models which are purely static.

\notfinished{}

\cite{ref:zamanipour2018a}

In this paper, a dynamic, axially-stiff string model is presented for force calculations of drillstring in tripping operations.

Uses lumped mass and spring system.

Claims to be an axial stiff string model.  Allows for the string to be at the top or bottom of the hole automatically.  It does not seem to account for the no contact case.
Uses a 2D wellbore.

Static friction and drilling fluid drag are accounted for.

Input a top drive axial velocity and predict surface displacement, bottom velocity, bottom displacement, and hook load.

Compares axial ``stiff string'' model to a ``soft string'' model.  It seems that ``dynamic'' and ``static'' would be better nomenclature for these models.

\section{ExxonMobil Dixit Model}

\notfinished{}
